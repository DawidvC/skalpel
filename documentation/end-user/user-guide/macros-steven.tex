%%%%%%%%%%%%%%%%%%%%%%%%%%%%%%%%%%%%%%%%%%%%%%%%%%%%%%%%%%%%%%%
%%%%%%%%%%%%%%%%%%%%%%%%%%%%%%%%%%%%%%%%%%%%%%%%%%%%%%%%%%%%%%%
%%
%% Copyright 2009, 2010 Steven Shiells
%%
%% This file is free software: you can redistribute it and/or modify
%% it under the terms of the GNU General Public License as published by
%% the Free Software Foundation, either version 3 of the License, or
%% (at your option) any later version.
%%
%% This file is distributed in the hope that it will be useful,
%% but WITHOUT ANY WARRANTY; without even the implied warranty of
%% MERCHANTABILITY or FITNESS FOR A PARTICULAR PURPOSE.  See the
%% GNU General Public License for more details.
%%
%% You should have received a copy of the GNU General Public License
%% along with Skalpel.  If not, see <http://www.gnu.org/licenses/>.
%%
%%
%% Authors: Steven Shiells
%% Date: December 2009
%% Description: My macros file
%%
%%%%%%%%%%%%%%%%%%%%%%%%%%%%%%%%%%%%%%%%%%%%%%%%%%%%%%%%%%%%%%%%
%%%%%%%%%%%%%%%%%%%%%%%%%%%%%%%%%%%%%%%%%%%%%%%%%%%%%%%%%%%%%%%%


\usepackage{color}

% my colors

\definecolor{mygray}{gray}{0.4}
\definecolor{fkgray}{gray}{0.2}
\definecolor{dkgray}{gray}{0.1}
\definecolor{ltgray}{cmyk}{0, 0, 0, 0.11}


\ifincolor
\definecolor{mywhite}{cmyk}{0.01, 0, 0.06, 0}
\definecolor{myyellow}{cmyk}{0, 0, 1, 0}
\definecolor{myred}{cmyk}{0, 0.443, 0.443, 0.0}
\definecolor{myorange}{cmyk}{0, 0.5, 0.8, 0.0}
\definecolor{myred2}{cmyk}{0, 0.9, 0.7, 0.1}
\definecolor{myblue}{cmyk}{0.4, 0.4, 0, 0}
\definecolor{mygreen}{cmyk}{1, 0, 1, 0}
\definecolor{mypurple}{cmyk}{0, 0.276, 0, 0.133}
\definecolor{mygrey}{cmyk}{0, 0, 0, 0.255}
\definecolor{inboxcol}{cmyk}{0,0,0,0}
\definecolor{mypink}{cmyk}{0,0.102,0.102,0}
\definecolor{mylightblue}{cmyk}{0.112,0.112,0,0}
\definecolor{mylightgreen}{cmyk}{0.122, 0, 0.122, 0}
\definecolor{mylightpurple}{cmyk}{0.008, 0.127, 0, 0.016}
\else
\definecolor{mywhite}{cmyk}{0, 0, 0, 0.05}
\definecolor{myyellow}{cmyk}{0, 0, 0, 0.1}
\definecolor{myred}{cmyk}{0, 0, 0, 0.2}
\definecolor{myorange}{cmyk}{0, 0, 0, 0.28}
\definecolor{myblue}{cmyk}{0, 0, 0, 0.35}
\definecolor{mygreen}{cmyk}{0, 0, 0, 0.5}
\definecolor{mypurple}{cmyk}{0, 0, 0, 0.41}
\definecolor{inboxcol}{cmyk}{0,0,0,0}
\fi

\newcommand{\whitetext}[1]{\textcolor{mywhite}{#1}}
\newcommand{\blacktext}[1]{\textcolor{black}{#1}}
\newcommand{\redtext}[1]{\textcolor{myred}{#1}}
\newcommand{\bluetext}[1]{\ifincolor
                \textcolor{blue}{#1}
        \else
                \textcolor{myblue}{#1}
        \fi}
\newcommand{\orangetext}[1]{\textcolor{myorange}{#1}}
\newcommand{\purpletext}[1]{\textcolor{mypurple}{#1}}
\newcommand{\greentext}[1]{\textcolor{mygreen}{#1}}
\newcommand{\yellowtext}[1]{\textcolor{myyellow}{#1}}

% commands for comments

\ifcomm
  \newcommand{\personal}[1]{}
  \ifincolor
  \newcommand{\todo}[2]{
    \noindent
    \begin{footnotesize}
      \textcolor{blue}{
        \textbf{
          [\begin{tiny}(#1)\end{tiny}: #2]}
      }
    \end{footnotesize}
  }
  \newcommand{\done}[2]{
    \noindent
    \begin{footnotesize}
      \textcolor{mygreen}{
        \textbf{
          [\begin{tiny}(#1)\end{tiny}: #2]}
      }
    \end{footnotesize}
  }
  \newcommand{\urgent}[2]{
    \noindent
    \begin{footnotesize}
      \textcolor{red}{
        \textbf{
          [\begin{tiny}(#1)\end{tiny}: #2]}
      }
    \end{footnotesize}
  }
 \newcommand{\comment}[2]{
    \noindent
    \begin{footnotesize}
      \textcolor{myorange}{
        \textbf{
          [\begin{tiny}(#1)\end{tiny}: #2]}
      }
    \end{footnotesize}
  }
  \else
  \newcommand{\todo}[2]{
    \noindent
    \begin{footnotesize}
      \textcolor{mygray}{
        \textbf{
          [\begin{tiny}(#1)\end{tiny}: #2]}
      }
    \end{footnotesize}
  }
 \newcommand{\done}[2]{
    \noindent
    \begin{footnotesize}
      \textcolor{fkgray}{
        \textbf{
          [\begin{tiny}(#1)\end{tiny}: #2]}
      }
    \end{footnotesize}
  }
  \newcommand{\urgent}[2]{
    \noindent
    \begin{footnotesize}
      \textcolor{dkgray}{
        \textbf{
          [\begin{tiny}(#1)\end{tiny}: #2]}
      }
      \end{footnotesize}
  }
  \newcommand{\comment}[2]{
    \noindent
    \begin{footnotesize}
      \textcolor{dkgray}{
        \textbf{
          [\begin{tiny}(#1)\end{tiny}: #2]}
      }
    \end{footnotesize}
  }
  \fi
\else
\newcommand{\personal}[1]{}
\newcommand{\todo}[2]{}
\newcommand{\done}[2]{}
\newcommand{\urgent}[2]{}
\newcommand{\comment}[2]{}
\fi

\newcommand{\codebody}[1]{{\tt\small\renewcommand{\arraystretch}{1}$$#1$$}}
\newcommand{\incodebody}[1]{{\tt\small\renewcommand{\arraystretch}{0.8}$\Bl#1\El$}}
\newcommand{\Bl}{\begin{tabular}[t]{@{}l@{}}}
\newcommand{\El}{\end{tabular}}
\newcommand{\Bi}{\begin{tabular}[t]{@{\quad}l}}
\newcommand{\Ei}{\end{tabular}}
\newcommand{\modifyboxdimen}[2]
  {% trashes \dimen0
   \dimen0=#1%
   \advance\dimen0 by #2%
   #1=\dimen0 % <- space is significant
  }

\newcommand{\examplebox}[2]
  {\begingroup
     \setbox0=\hbox{(g)}% use to determine standard height and depth
     \modifyboxdimen{\ht0}{0.5pt}%
     \modifyboxdimen{\dp0}{0pt}%
     \setbox1=\hbox{#2}%
     \ht1=\ht0%
     \dp1=\dp0%
     \setlength{\fboxsep}{0pt}% default 3pt
     \setbox2=\hbox{%\TraceExec
                    \colorbox{#1}{\box1}}%
                    % \colorbox{#1}{{\color{black}\box1}}}% only for talk
     %\bgroup
     % \tracingonline=1\relax
     % \showboxdepth=9999\relax
     % \showboxbreadth=9999\relax
     % \showbox2\relax
     %\egroup
     \modifyboxdimen{\ht2}{-0.5pt}%
     \modifyboxdimen{\dp2}{-0pt}%
     \box2%
   \endgroup}
\newcommand{\examplefbox}[2]
  {\begingroup
     \setbox0=\hbox{(g)}% use to determine standard height and depth
     \modifyboxdimen{\ht0}{0.5pt}%
     \modifyboxdimen{\dp0}{0pt}%
     \setbox1=\hbox{#2}%
     \ht1=\ht0%
     \dp1=\dp0%
     \setlength{\fboxrule}{1pt}% for talk only
     \setlength{\fboxsep}{0pt}% default 3pt
     \setbox2=\hbox{%\TraceExec
                    \fcolorbox{#1}{inboxcol}{\box1}}%
     %\bgroup
     % \tracingonline=1\relax
     % \showboxdepth=9999\relax
     % \showboxbreadth=9999\relax
     % \showbox2\relax
     %\egroup
     \modifyboxdimen{\ht2}{-0.5pt}%
     \modifyboxdimen{\dp2}{-0pt}%
     \box2%
   \endgroup}
\newcommand{\exampleffbox}[3]
  {\begingroup
     \setbox0=\hbox{(g)}% use to determine standard height and depth
     \modifyboxdimen{\ht0}{0.5pt}%
     \modifyboxdimen{\dp0}{0pt}%
     \setbox1=\hbox{#2}%
     \ht1=\ht0%
     \dp1=\dp0%
     \setlength{\fboxrule}{1pt}% for talk only
     \setlength{\fboxsep}{0pt}% default 3pt
     \setbox2=\hbox{%\TraceExec
                    \fcolorbox{#1}{#3}{\box1}}%
     %\bgroup
     % \tracingonline=1\relax
     % \showboxdepth=9999\relax
     % \showboxbreadth=9999\relax
     % \showbox2\relax
     %\egroup
     \modifyboxdimen{\ht2}{-0.5pt}%
     \modifyboxdimen{\dp2}{-0pt}%
     \box2%
   \endgroup}

\newcommand{\boxBl}[1]{\examplebox{black}{#1}}
\newcommand{\boxW}[1]{\examplebox{mywhite}{#1}}
\newcommand{\boxY}[1]{\examplebox{myyellow}{#1}}
\newcommand{\boxR}[1]{\examplebox{myred}{#1}}
\newcommand{\boxO}[1]{\examplebox{myorange}{#1}}
\newcommand{\boxB}[1]{\examplebox{myblue}{#1}}
\newcommand{\boxG}[1]{\examplebox{mygreen}{#1}}
\newcommand{\boxP}[1]{\examplebox{mypurple}{#1}}
\newcommand{\boxEnd}[1]{\examplebox{mygrey}{#1}}
\newcommand{\fboxBl}[1]{\examplefbox{black}{#1}}
\newcommand{\fboxW}[1]{\examplefbox{mywhite}{#1}}
\newcommand{\fboxY}[1]{\examplefbox{myyellow}{#1}}
\newcommand{\fboxR}[1]{\examplefbox{myred}{#1}}
\newcommand{\fboxB}[1]{\examplefbox{myblue}{#1}}
\newcommand{\fboxG}[1]{\examplefbox{mygreen}{#1}}
\newcommand{\fboxO}[1]{\examplefbox{myorange}{#1}}
\newcommand{\fboxP}[1]{\examplefbox{mypurple}{#1}}
\newcommand{\fboxEnd}[1]{\examplefbox{mygrey}{#1}}
\newcommand{\fboxGen}[3]{\exampleffbox{#1}{#2}{#3}} %% #1 colour outside, #2 text, #3 colour inside

\newcommand{\instruction}[1]{\incodebody{<{#1}>}}

\newcommand{\tesEndPointOne}[0]{gray }

\newcommand{\note}[1]{\medskip\textbf{NOTE:} {#1}}

\newcommand{\tesDownloadURL}[0]{\codebody{\Bl http://www2.macs.hw.ac.uk/{\textasciitilde}rahli/cgi-bin/slicer/downloads.html \El}}
\newcommand{\tes}[0]{type error slicer}
\newcommand{\tesing}[0]{type error slicing}
\newcommand{\Tesing}[0]{Type Error Slicing}
\newcommand{\Tes}[0]{Type Error Slicer}
\newcommand{\file}[1]{\textsf{#1}}
