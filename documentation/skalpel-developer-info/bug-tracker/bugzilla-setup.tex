\documentclass[11pt, a4paper]{report}%twopage and openright need removed for single sided page printing
\usepackage[left=92px, right=92px, top=42px]{geometry}
\usepackage[table, svgnames]{xcolor}
\usepackage[pdftex]{graphicx}
\usepackage{subfig}
\usepackage{listings}
\usepackage{setspace}
\usepackage{url}
\usepackage{color}
\definecolor{directories}{cmyk}{0.56,0.48,0.14,0.1}
\definecolor{links}{cmyk}{0.79,0.69,0.59,0.59}
\definecolor{nonuniquedirectory}{cmyk}{0.42,0.71,0,0}
\definecolor{reference}{cmyk}{0.55,0.69,0,0}
\newcommand{\tab}{\hspace*{2em}}
\usepackage{hyperref}
\hypersetup{
    bookmarks=true,         % show bookmarks bar?
    unicode=false,          % non-Latin characters in Acrobat’s bookmarks
    pdftoolbar=true,        % show Acrobat’s toolbar?
    pdfmenubar=true,        % show Acrobat’s menu?
    pdffitwindow=false,     % window fit to page when opened
    pdfstartview={FitH},    % fits the width of the page to the window
    pdftitle={My title},    % title
    pdfauthor={Author},     % author
    pdfsubject={Subject},   % subject of the document
    pdfcreator={Creator},   % creator of the document
    pdfproducer={Producer}, % producer of the document
    pdfkeywords={keyword1} {key2} {key3}, % list of keywords
    pdfnewwindow=true,      % links in new window
	breaklinks=true,		% linebreak long urls
    colorlinks=true,        % false: boxed links; true: colored links
    linkcolor=links,        % color of internal links
	linkbordercolor=brown,	% color border links this color if they are black
    citecolor=reference,    % color of links to bibliography
    filecolor=magenta,      % color of file links
    urlcolor=cyan           % color of external links
}
\newcommand{\tes}[0]{\textbf{TES}}
\newcommand{\tesplain}[0]{TES}
\newcommand{\tesing}[0]{type error slicing}
\newcommand{\smltes}[0]{sml-tes}
\renewcommand{\thesubfigure}{\arabic{subfigure}}
\begin{document}
\parindent=0px
\title{Setting Up a Bugzilla Installation for The Skalpel Project}
\author{Scott Fotheringham}
\date{20.08.2011}
\maketitle
\vspace{110mm}
\small{
\noindent \textcopyright 2010 Scott Fotheringham
\\
Permission is granted to copy, distribute and/or modify this document
under the terms of the GNU Free Documentation License, Version 1.3 or
any later version published by the Free Software Foundation; with no
Invariant Sections, no Front-Cover Texts, and no Back-Cover Texts.  A
copy of the license is included in the section entitled "GNU Free
Documentation License".}

\newenvironment{changemargin}[2]{
	\begin{list}{}{\setlength{\leftmargin}{#1}
				   \setlength{\rightmargin}{#2}}
				   \item[]}{
	\end{list}}

\onehalfspacing

\tableofcontents

% --- --- --- --- --- --- --- --- --- --- --- --- --- --- --- --- --- --- --- --- --- --- --- --- --- --- --- --- --- --- --- --- --- --- --- Chapter --- %
\chapter{Installation}
\label{ch:install}
This chapter documents getting, installing and configuring a fresh Bugzilla installation up until the point where it is working with a default configuration.
For the purposes of this document, the current stable version of Bugzilla is 4.0.2 and will be used throughout. This document should be fairly future-proof but may need to be updated.

% --- --- --- --- --- --- --- --- --- --- --- --- --- --- --- --- --- --- --- --- --- --- --- --- --- --- --- --- --- --- --- --- --- --- --- Section --- %
\section{Getting Bugzilla}
\label{sec:getbugzilla}
Log into the www server, Athena, and navigate to\\\emph{/var/www/html/macs/ultra/compositional-analysis/type-error-slicing}.\\\\
Using the command {\tt wget}, download the latest version of Bugzilla.\\For example:\\{\tt wget http://ftp.mozilla.org/pub/mozilla.org/webtools/bugzilla-4.0.2.tar.gz}\\
\\
Extract the compressed package.\\For example:\\{\tt tar -zxvf bugzilla-4.0.2.tar.gz}


% --- --- --- --- --- --- --- --- --- --- --- --- --- --- --- --- --- --- --- --- --- --- --- --- --- --- --- --- --- --- --- --- --- --- --- Section --- %
\section{Initial Setup}
\label{sec:initsetup}
Navigate to the root of the extracted directory and run the command\\
{\tt ./checksetup.pl --check-modules}\\
which will check for Perl module dependencies. Most likely, there will be no need to install extra modules. However, if there is, run the command\\
{\tt perl install-module.pl --all}\\
and have a nap because this takes a long time.
\newpage
Once this completes, run\\
{\tt ./checksetup.pl --check-modules}\\
again to ensure everything is cushty and finally, run\\
{\tt ./checksetup.pl}.\\
\\
A file, \emph{localconfig}, is created. This file links the database to the new Bugzilla installation. The variables in this file should be set to the following:\\
\\
create\_htaccess = 0\\
webservergroup = `www'\\
use\_suexec = 1\\
db\_driver = `mysql'\\
db\_host = `mysql-server-1'\\
db\_name = `ultra'\\
db\_user = `ultra'\\
db\_pass = `ultra'\\
db\_port = 0\\
db\_check = 1\\
index\_html = 0\\\\
All other variables should be left at their default values.

% --- --- --- --- --- --- --- --- --- --- --- --- --- --- --- --- --- --- --- --- --- --- --- --- --- --- --- --- --- --- --- --- --- --- --- Section --- %
\section{Backing Up The Database (Because Shit Happens)}
\label{sec:backupdb}
Before proceeding, ensure the existing database is backed up using the command {\tt mysqldump -h mysql-server-1 -u ultra -pultra ultra > database-backup-[date].sql}


% --- --- --- --- --- --- --- --- --- --- --- --- --- --- --- --- --- --- --- --- --- --- --- --- --- --- --- --- --- --- --- --- --- --- --- Section --- %
\section{Finishing Installation}
\label{sec:fininstall}
At this point, run\\
{\tt ./checksetup.pl}\\
one last time to propogate changes made to \emph{localconfig}


% --- --- --- --- --- --- --- --- --- --- --- --- --- --- --- --- --- --- --- --- --- --- --- --- --- --- --- --- --- --- --- --- --- --- --- Section --- %
\section{Making Bugzilla Work}
\label{sec:makework}
{\bf This section of the document will only have to stay here as long as IT support are too paranoid to run checksetup.pl as root}
\\
\\
Create a file in the root directory of the Bugzilla installation called \emph{index.php}. This file should contain exactly\\
\\
\emph{\textless?php header("location:index.cgi"); ?\textgreater}\\

This points visitors to index.cgi because our web server is crap.\\
\\
Finally, run the commands\\
{\tt setfacl -R -m www:rwx bugzilla-4.0.2}\\
and\\
{\tt chmod -R 775 bugzilla-4.0.2}\\
to allow permissions to the web server to write to the files it needs to and create temporary files as it needs.


% --- --- --- --- --- --- --- --- --- --- --- --- --- --- --- --- --- --- --- --- --- --- --- --- --- --- --- --- --- --- --- --- --- --- --- Section --- %
\section{Finish Setup}
\label{sec:finsetup}
Now you can go to the HTTP URL for the Bugzilla installation. This will be something like\\
\emph{www.macs.hw.ac.uk/ultra/compositional-analysis/type-error-slicing/testbug/bugzilla-4.0.2}\\
where you will log in and use the Administration area to set the correct HTTP URL base and cookie path.\\
\\
For the above URL, the URL base would be\\
\emph{http://www.macs.hw.ac.uk/ultra/compositional-analysis/type-error-slicing/testbug/}\\
and cookiepath would be\\
\emph{/ultra/compositional-analysis/type-error-slicing/testbug}

% --- --- --- --- --- --- --- --- --- --- --- --- --- --- --- --- --- --- --- --- --- --- --- --- --- --- --- --- --- --- --- --- --- --- --- Chapter --- %
\chapter{Customizing Bugzilla}
\label{ch:custombugzilla}
This chapter documents customizing the Bugzilla installation for The Skalpel Project.


% --- --- --- --- --- --- --- --- --- --- --- --- --- --- --- --- --- --- --- --- --- --- --- --- --- --- --- --- --- --- --- --- --- --- --- Section --- %
\section{Customizing The Template}
\label{sec:customtemplate}
Navigate (from the root directory) to\\
\emph{templates/en/default/global}\\
and open \emph{variables.none.tmpl}.\\
\\
In this file, change the phrase\\
``Zaroo Boogs Found'' to ``No bugs found matching search criteria''.\\
Also in this file, change the line\\
\emph{``Bugzilla'' =\textgreater ``Bugzilla''} to \emph{``Bugzilla'' =\textgreater ``Skalpel Bug Tracker''}\\
\\
After any changes to template files, it is necessary to run\\
{\tt ./checksetup.pl}\\
in the root directory to propogate changes made.

\end{document}
